\documentclass[nolof,digital]{fithesis3}
\thesissetup{faculty=fi}
\thesissetup{author=Bc. Lukáš Tyrychtr,id=422294, departmentEn = Faculty of Informatics, programmeEn = Applied Informatics, fieldEn = Applied Informatics, assignment = {}}
\thesissetup{type= mgr}
\thesissetup{title=Feel the streets - a visually impaired user access to OSM maps}
\thesissetup{keywords = {visually impaired, navigation, accessibility, Open Street Map, usability}}
\usepackage[english]{babel}
\usepackage{alltt}
\usepackage{listings}
\usepackage{csquotes}
\usepackage[hidelinks]{hyperref}
\usepackage[style=ieee,sorting=nty,block=ragged]{biblatex}
\renewbibmacro*{bbx:savehash}{} %disabling dashing
\addbibresource{bibliography.bib}
\usepackage{tabularx}
\usepackage{placeins}
\usepackage{tabu}
\usepackage{algorithm2e}
\SetAlCapSkip{3mm}
\thesislong{abstract}{

}
\thesissetup{advisor=Ing. Milan Brož\, Ph.D.}
\thesislong{thanks}{
}
\thesisload
\setcounter{tocdepth}{2}
\begin{document}
\chapter{Introduction}
\chapter{Map presentation approaches}
This chapter describes the various approaches which were used to present maps to visually impaired people, mainly focusing on tactile maps and various application for smart phones and desktop operating systems.
\section{Printed maps}
This section describes the usage of printed maps.
\section{Mobile phone apps}
This section explores the various mobile phone apps used for navigation by visually impaired people as well as the usability of standard map applications on the respective mobile phone operating systems.
\section{Desktop apps}
In this section, the few desktop applications for navigation purposes are explored.
\chapter{Feel the streets}
This chapter describes the implementation of Feel the streets, an application for browsing Open Street Maps data by visually impaired people, its features, implementation decisions and some noteworthy events in the pplication development.
\section{Feature overview}
This section describes the currently implemented features of Feel the Streets.
\section{Technical overview}
This section describes the currently used technologies which are used by the Feel the Streets client and server. The technologies have changed quite extensively during the development, so the Development history section then presents the various choices which were made during the lifetime of the application.
\section{Development history}
This section presents the development history of Feel the Streets, the various approaches used and their advantages and disadvantages. It also mentions some of the accessibility issues which are further described in the Issuses encountered along the way section.
\section{Issues encountered along the way}
This section describes the various issues encountered during the development of Feel the Streets. Many of them were related to the OSM data or the data model, but there were some accessibility relaed issues as well.
\subsection{Representing crossings}
\subsection{Footways}
\subsection{Accessibility issues}
\chapter{Usability study}
This chapter presents the study which was carried between visually impaired users concerning the usability of Feel the streets
\section{Study description}
This section briefly describes the study and its goals
\section{About the subjects}
This section describes the subjects and general reasons for the particular selection.
\section{Methodology}
This section describes the methodology of the study and gives a reasoning for choosing the particular methods.
\section{The questions and the answers}
In this section, the questions are presented, along with the summarized answers.
\chapter{Conclusion}
\section{In summary}
\section{Future work}
\end{document}